% communication.tex
% Content for the CM101 - Communication Skills (Writing & Speaking) module.

\section{CM101 - Communication Skills (Writing \& Speaking)}
\addtocounter{section}{2} % Ensure section counter is correct if needed for auto-numbering consistency.

\begin{tcolorbox}[boxstyle, title={Module Overview}]
    This module equips students with essential fundamental speaking and writing skills, enabling them to communicate effectively in various contexts. It focuses on building confidence in self-expression, engaging in meaningful conversations, structuring ideas for both oral presentations and written texts, and refining their work through revision and editing.
\end{tcolorbox}

\subsection{S1: Speaking – Basic Introductions \& Everyday Conversation}
\objectivebox{
    \item \textbf{Confidently introduce} themselves and others using basic English phrases.
    \item \textbf{Engage} in simple everyday conversations (e.g., asking about name, origin, simple preferences).
    \item \textbf{Apply} basic politeness expressions in conversation.
}
\materialsbox{
    \item \textbf{Teaching Aids:} Whiteboard/markers, projector, flashcards with common greetings and questions, audio clips of basic conversations, handouts with dialogue examples.
}
\lecturebox{
    \begin{itemize}
        \item \textbf{Basics of Greetings and Introductions:} Formal and informal greetings, introducing oneself (name, age, origin/background) and others.
        \item \textbf{Simple Questions and Answers for Daily Interactions:} Covering topics like "How are you?", "What do you do?", "Where are you from?".
        \item \textbf{Pronunciation Focus on Key Phrases:} Emphasizing clarity and common stress patterns in conversational English.
        \item \textbf{Politeness and Common Expressions:} Incorporating "please," "thank you," "excuse me," "I'm sorry," and "you're welcome."
    \end{itemize}
}
\interactivebox{
    \begin{itemize}
        \item \textbf{Pair Practice (60 minutes):} Role-playing self-introductions and greetings in various scenarios (e.g., meeting a new classmate, introducing a friend).
        \item \textbf{Listening Exercises (20 minutes):} Students listen to audio clips of basic conversations and answer comprehension questions, then practice repeating key phrases.
        \item \textbf{Group Activities (10 minutes):} "Find someone who\dots" game using simple questions (e.g., "Find someone who likes to read," "Find someone who has a pet").
        \item \textbf{Instructor Feedback (30 minutes):} Individualized feedback on pronunciation, intonation, and overall conversational fluency.
        \item \textbf{Assessment:} Active participation in role-plays, ability to form complete sentences in introductions and answers, observed confidence.
    \end{itemize}
}
\subsection{\texorpdfstring{S2: Speaking
\\Topic-Based Conversations }
                              {S2: Speaking – Topic-Based Conversations \& Fluency}}

%\subsection{S2: Speaking – Topic-Based Conversations \& Fluency}
\objectivebox{
    \item \textbf{Discuss} familiar topics (e.g., family, school, hobbies) using simple sentences with increased clarity and fluency.
    \item \textbf{Utilize} techniques to improve speaking fluency, such as linking words and appropriate pacing.
    \item \textbf{Express} basic opinions and preferences politely.
}
\materialsbox{
    \item \textbf{Teaching Aids:} Topic cards (e.g., "My Family," "My School Day," "My Hobbies," "Future Dreams"), pronunciation drill worksheets, audio recordings of sample dialogues.
}
\lecturebox{
    \begin{itemize}
        \item \textbf{Vocabulary and Sentence Structures for Everyday Topics:} Expanding vocabulary related to common themes and practicing forming descriptive sentences.
        \item \textbf{Fluency Techniques:} Introduction to linking words (e.g., "I wanna go," "gonna"), pausing for effect, and maintaining a natural conversational pace.
        \item \textbf{Pronunciation Drills:} Focusing on common problem sounds for the students' native language background, intonation patterns for questions and statements.
       \item  \textbf{Introduction to Expressing Opinions Politely:} Phrases like "I think...", "In my opinion...", "I agree/disagree because...".
    \end{itemize}
}
\interactivebox{
    \begin{itemize}
        \item \textbf{Small Group Discussions (60 minutes):} Students pick topic cards and engage in guided discussions, using the vocabulary and sentence structures introduced.
        \item \textbf{Fluency Games and Tongue Twisters (30 minutes):} Activities designed to improve articulation and speaking rhythm, such as repeating challenging phrases or simple tongue twisters.
        \item \textbf{Roleplay Scenarios (20 minutes):} Practicing asking and answering questions about opinions and preferences (e.g., "What's your favorite type of music and why?").
        \item \textbf{Peer and Instructor Feedback (10 minutes):} Feedback focused on clarity of expression, use of new vocabulary, and improvements in fluency.
        \item \textbf{Assessment:} Participation in discussions, observed improvement in fluency and pronunciation drills, ability to express simple opinions.
    \end{itemize}
}
\subsection{\texorpdfstring{S3: Speaking – Short \\ Presentations \& Group Discussions}
                              {S3: Speaking – Short Presentations \& Group Discussions}}

%\subsection{S3: Speaking – Short \\Presentations \& Group Discussions}
\objectivebox{
    \item \textbf{Prepare and deliver} a short, organized presentation on a familiar topic.
    \item \textbf{Participate} actively and respectfully in group discussions, demonstrating turn-taking skills.
    \item \textbf{Apply} foundational public speaking skills (eye contact, voice control, body language) in a presentation setting.
}
\materialsbox{
    \item \textbf{Teaching Aids:} Note cards for presentations, presentation checklist handouts, video examples of short speeches, timers for practice.
}
\lecturebox{
    \begin{itemize}
        \item \textbf{Structure of a Short Presentation:} Introduction (hook, topic, roadmap), Main Points (supported by simple details), Conclusion (summary, takeaway).
        \item \textbf{Tips for Eye Contact, Voice Control, and Body Language:} Practical advice and exercises for maintaining engagement, varying voice tone/volume, and using gestures effectively.
       \item \textbf{Basics of Group Discussion Etiquette:} Active listening in a group, turn-taking signals, agreeing/disagreeing politely, staying on topic.
    \end{itemize}
}
\interactivebox{
    \begin{itemize}
        \item \textbf{Presentation Preparation (60 minutes):} Students prepare a 2-3 minute speech on a familiar topic (e.g., "My Family," "My Favorite Place," "My Dream Job") using note cards and the provided checklist.
        \item \textbf{Deliver Presentations (45 minutes):} Students deliver their presentations to small groups (3-4 students). Each presentation is followed by a brief Q\&A from the group.
        \item \textbf{Group Discussion Exercises (15 minutes):} Facilitated discussions on guided questions where students practice active listening and respectful turn-taking.
        \item \textbf{Constructive Feedback (20 minutes):} Peers and the instructor provide specific, actionable feedback on presentation structure, delivery elements (eye contact, voice), and participation in discussion.
        \item \textbf{Assessment:} Delivery of a structured short presentation, demonstrated ability to participate constructively in group discussions, application of basic public speaking techniques.
    \end{itemize}
}

\subsection{\texorpdfstring{S4: Writing – Sentence \\ Structure and Basic Paragraphs}
                              {S4: Writing – Sentence Structure and Basic Paragraphs}}
%\subsection{S4: Writing – Sentence \\ Structure and Basic Paragraphs}
\objectivebox{
    \item \textbf{Identify} the components of a simple sentence (subject, verb, object).
    \item \textbf{Construct} grammatically correct simple and compound sentences.
    \item \textbf{Understand and apply} the basic structure of a paragraph (topic sentence, supporting details, concluding sentence).
    \item \textbf{Utilize} common punctuation rules (periods, commas, capitalization).
}
\materialsbox{
    \item \textbf{Teaching Aids:} Sentence structure charts, paragraph writing worksheets, sample sentences and paragraphs for analysis, simple grammar rule handouts.
}
\lecturebox{
    \begin{itemize}
        \item \textbf{Parts of Speech Review:} Focus on nouns, verbs, and adjectives and their roles in sentences.
        \item \textbf{Constructing Simple and Compound Sentences:} Explanation of subject-verb agreement. Introduction to coordinating conjunctions (FANBOYS: For, And, Nor, But, Or, Yet, So) for compound sentences.
        \item \textbf{Paragraph Structure:} Introduction to the concept of a main idea (topic sentence) and how supporting details develop that idea. Discussion of concluding sentences.
        \item \textbf{Common Punctuation Rules:} Emphasis on periods for sentence endings, commas in lists and compound sentences, and capitalization for proper nouns and sentence beginnings.
    \end{itemize}
}
\interactivebox{
    \begin{itemize}
        \item \textbf{Sentence Building Exercises (45 minutes):} Students work with word cards (or digital equivalents) to build correct simple and compound sentences. Exercises to identify subjects, verbs, and objects in given sentences.
        \item \textbf{Writing Simple Sentences and Paragraphs (35 minutes):} Students practice writing a series of simple sentences on a given topic, then combine them into a coherent short paragraph (5-7 sentences) with a clear topic sentence.
        \item \textbf{Peer Review and Instructor Guidance (25 minutes):} Students exchange paragraphs for peer review, focusing on sentence clarity, grammar, punctuation, and paragraph structure. Instructor provides targeted feedback.
        \item \textbf{Writing Prompts (35 minutes):} Quick, guided writing prompts for immediate paragraph practice (e.g., "Describe your favorite animal," "What did you do last weekend?").
        \item \textbf{Assessment:} Correct formation of simple/compound sentences, ability to write a structured paragraph, demonstrated understanding of basic punctuation.
    \end{itemize}
}

\subsection{\texorpdfstring{S5: Writing – Writing for \\ Real Life}
                              {S5: Writing – Writing for Real Life}}

%\subsection{S5: Writing – Writing for \\ Real Life }
\objectivebox{
    \item \textbf{Compose} short, practical texts for real-life situations, such as informal emails and friendly letters.
    \item \textbf{Outline} the fundamental structure of a simple essay (introduction, body paragraphs, conclusion).
    \item \textbf{Employ} basic linking words and transition phrases to improve text cohesion.
    \item \textbf{Identify} and avoid common grammatical pitfalls in extended writing.
}
\materialsbox{
    \item \textbf{Teaching Aids:} Sample informal emails and friendly letters, simple essay outline templates, writing prompt handouts focusing on opinion-based topics.
}
\lecturebox{
    \begin{itemize}
        \item \textbf{Format and Language for Informal Emails and Letters:} Key components (salutation, body, closing), appropriate tone, common abbreviations, and expressions.
        \item \textbf{Introduction to Essay Structure:} Detailed breakdown of the three main parts of a basic essay:
            \begin{itemize}
                \item \textbf{Introduction:} Hook, background, thesis statement.
                \item \textbf{Body Paragraphs:} Topic sentence, supporting details/examples, concluding sentence.
                \item \textbf{Conclusion:} Restate thesis, summarize main points, final thought.
            \end{itemize}
       \item \textbf{Linking Words and Transition Phrases:} Introduction to words and phrases that connect ideas and paragraphs (e.g., "first," "also," "however," "in conclusion").
        \item \textbf{Common Grammar Pitfalls in Writing:} Review of frequent errors like run-on sentences, sentence fragments, and subject-verb agreement in longer texts.
    \end{itemize}
}
\interactivebox{
    \begin{itemize}
        \item \textbf{Practical Writing Practice (40 minutes):} Students draft a short informal email to a friend or family member, or a friendly letter inviting someone to an event. They will then peer-edit each other's drafts for clarity, tone, and appropriate language.
        \item \textbf{Essay Brainstorming and Outlining (45 minutes):} Students choose from a list of familiar topics (e.g., "The importance of hobbies," "My favorite season," "Should students have homework?"). They will brainstorm ideas and create a detailed outline for a simple 3-paragraph essay (intro, one body, conclusion) using the provided templates.
        \textbf{Group Sharing of Essay Ideas and Outlines (25 minutes):} Students share their essay outlines and topic sentences with the class for initial feedback on organization and clarity.
        \item \textbf{Instructor Feedback (20 minutes):} Feedback on the structure of drafted practical texts and the logical flow of essay outlines.
       \item \textbf{Assessment:} Completion of a well-structured informal email/letter, logical and complete essay outline, demonstrated use of linking words.
    \end{itemize}
}
\subsection{\texorpdfstring{S6: Revision\\ Editing \& Creative Writing}
                              {S6: Revision Editing \& Creative Writing}}

%\subsection{S6: Revision\\ Editing \& Creative Writing}
\objectivebox{
    \item \textbf{Apply} systematic revision strategies to improve the content, organization, and clarity of their written work.
    \item \textbf{Utilize} editing checklists to identify and correct common grammatical, spelling, and punctuation errors.
    \item \textbf{Experiment} with basic creative writing techniques (e.g., descriptive language, imagery) in short tasks.
    \item \textbf{Reflect} on their writing process and identify areas for personal improvement.
}
\materialsbox{
    \item \textbf{Teaching Aids:} Editing checklists (e.g., for grammar, spelling, punctuation, clarity), sample writing with intentional errors for correction, creative writing prompts (e.g., picture prompts, opening lines), vocabulary building lists for descriptive writing.
}
\lecturebox{
    \begin{itemize}
        \item \textbf{Importance of Revision and Self-Editing:} Why writing is a process, not a single event. Distinction between revision (content, organization) and editing (grammar, mechanics).
        \item \textbf{Common Writing Errors and How to Fix Them:} Review of persistent errors identified in previous sessions (e.g., fragments, run-ons, comma splices, common spelling mistakes). Strategies for identifying and correcting these.
        \item \textbf{Techniques for Creative Expression in Writing:} Show, don't tell; using sensory details; figurative language (similes, metaphors - simplified).
        \item \textbf{Vocabulary Building for Descriptive Writing:} Introduction to stronger verbs and more precise adjectives to enhance descriptive quality.
    \end{itemize}
}
\interactivebox{
    \begin{itemize}
        \item \textbf{Editing Practice (45 minutes):} Students work individually or in pairs to identify and correct errors in provided sample texts that contain common grammatical, spelling, and punctuation mistakes.
        \item \textbf{Revising and Improving Student Work (60 minutes):} Students apply revision and editing checklists to their own previously written paragraphs or essay outlines. They will focus on strengthening topic sentences, adding more supporting details, improving transitions, and correcting errors.
        \item \textbf{Sharing and Feedback (10 minutes):} Students share excerpts from their revised or creative writing in small groups, receiving feedback on clarity, impact, and areas for further refinement.
        \item \textbf{Assessment:} Demonstrated ability to revise and edit texts using checklists, application of creative writing techniques, thoughtful self-reflection on writing process.
    \end{itemize}
}
