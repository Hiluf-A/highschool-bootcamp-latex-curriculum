% preamble.tex
% This file contains all global settings, package inclusions, and custom commands for the document.

% Document Class
\documentclass[12pt, a4paper]{article} % Using article class for flexibility in styling. 11pt font, A4 paper size.

% --- PACKAGES ---
%\usepackage[utf8]{inputenc} % Allows direct input of UTF-8 characters. Ignore it if you are using LuaLaTeX or XeLaTeX.
\usepackage[T1]{fontenc}    % Optimizes font encoding for better output quality.

% Layout and Geometry
\usepackage[left=2.5cm, right=2.5cm, top=2.5cm, bottom=2.5cm]{geometry} % Sets page margins.

% Font Configuration
\usepackage{helvet} % Helvetica-like font (sans-serif)
\renewcommand{\familydefault}{\sfdefault} % Set sans-serif as default font.
\usepackage{lmodern} % Fallback for Latin Modern fonts, good for math.
\usepackage{microtype} % Improves text appearance (protrusion, expansion, etc.).

% Colors
\usepackage[table]{xcolor} % For custom colors.
\definecolor{primaryblue}{HTML}{0056B3}   % A deep blue for main headings, etc.
\definecolor{secondarygreen}{HTML}{28A745} % A vibrant green for highlights.
\definecolor{accentorange}{HTML}{FD7E14}  % An orange for subtle accents.
\definecolor{lightgray}{HTML}{F8F9FA}     % Light background for boxes.
\definecolor{darkgray}{HTML}{343A40}      % Dark text/lines.
\definecolor{boxborder}{HTML}{CED4DA}     % Border for boxes.
\definecolor{lessonboxbg}{HTML}{E9ECEF}   % 
\definecolor{seminargold}{HTML}{FFEFD5}  % Pale gold background
\definecolor{seminargreen}{HTML}{E6F5EB} % Mint green for seminar cells
\definecolor{seminartext}{HTML}{2E8B57}  % Forest green text 

% Hyperlinks
\usepackage[colorlinks, linkcolor=primaryblue, citecolor=secondarygreen, urlcolor=accentorange]{hyperref}
% Makes links clickable and colored.

% Graphics and Images
\usepackage{graphicx} % For including images.
\graphicspath{{images/}} % Specify path for images.

% Lists
\usepackage{enumitem} % Enhanced list environments.
\setlist[itemize]{label=\color{primaryblue}\textbullet, leftmargin=*, noitemsep} % Custom bullet color.
\setlist[enumerate]{label=\color{primaryblue}\arabic*., leftmargin=*, noitemsep} % Custom enumeration color.

% Headers and Footers
\usepackage{fancyhdr} % For custom headers and footers.
\pagestyle{fancy}
\fancyhf{} % Clear all header and footer fields.
\fancyhead[L]{\nouppercase{\rightmark}} % Section name in left header (e.g., "1. CS100 - Basic Computer Skills")
\fancyhead[R]{\nouppercase{\leftmark}} % Subsection name in right header (e.g., "S1: Intro to Computers & Operating Systems")
\fancyfoot[C]{\thepage} % Page number in center footer.
\renewcommand{\headrulewidth}{0.5pt} % Line under header.
\renewcommand{\footrulewidth}{0pt} % No line under footer.
\fancypagestyle{plain}{ % Redefine plain style (e.g., for first page)
  \fancyhead{} % No header on plain pages.
  \fancyfoot[C]{\thepage} % Page number only.
  \renewcommand{\headrulewidth}{0pt}
  \renewcommand{\footrulewidth}{0pt}
}

% Sectioning and Titles
\usepackage{titlesec} % Customizable section titles.
\titleformat{\section}{\Large\bfseries\color{primaryblue}}{\thesection.}{1em}{} % Style for sections.
\titleformat{\subsection}{\large\bfseries\color{secondarygreen}}{\thesubsection.}{1em}{} % Style for subsections.
\titleformat{\subsubsection}{\bfseries\color{darkgray}}{\thesubsubsection.}{1em}{} % Style for subsubsections.
\titlespacing*{\section}{0pt}{1.5ex plus 1ex minus .2ex}{1ex plus .2ex}
\titlespacing*{\subsection}{0pt}{1.2ex plus 1ex minus .2ex}{1ex plus .2ex}
\titlespacing*{\subsubsection}{0pt}{1ex plus 1ex minus .2ex}{1ex plus .2ex}

% Boxes for Content (Lecture, Interactive Session, Objectives, Materials)
\usepackage[most]{tcolorbox} % For creating colored boxes.

% Custom tcolorbox styles
\tcbset{
    boxstyle/.style={
        enhanced,
        colback=lessonboxbg,     % Background color for the box content
        colframe=boxborder,      % Frame color
        boxrule=0.4pt,           % Frame thickness
        arc=4pt,                 % Rounded corners
        outer arc=4pt,
        top=2mm, bottom=2mm, left=2mm, right=2mm, % Padding inside box
        parbox=false, % Ensure paragraphs are handled correctly
        breakable, % Allow boxes to break across pages
        % auto outer arc, % Automatically adjust outer arc based on arc
    },
    learningobjectives/.style={
        boxstyle,
        coltitle=white,           % Title color
        colbacktitle=primaryblue, % Title background color
        fonttitle=\bfseries,      % Title font style
        attach title to upper,    % Title is part of the upper part of the box
        title={Learning Objectives}, % Default title for this style
        left=2mm,right=2mm,top=2mm,bottom=2mm, % Internal padding
        % icon={%
        %     \includegraphics[height=1em]{icons/objective_icon.png}% You could use an image here
        % },
        % title code={\faIcon[regular]{check-circle} \phantomsection}, % FontAwesome icon for check-circle
    },
    materials/.style={
        boxstyle,
        coltitle=white,
        colbacktitle=secondarygreen,
        fonttitle=\bfseries,
        attach title to upper,
        title={Materials},
        % icon={\faIcon[solid]{box}}
    },
    lecture/.style={
        boxstyle,
        coltitle=white,
        colbacktitle=darkgray,
        fonttitle=\bfseries,
        attach title to upper,
        title={Lecture (1 hour)},
        % icon={\faIcon[solid]{chalkboard-teacher}}
    },
    interactive/.style={
        boxstyle,
        coltitle=white,
        colbacktitle=accentorange,
        fonttitle=\bfseries,
        attach title to upper,
        title={Interactive Session (2 hours)},
        % icon={\faIcon[solid]{users}}
    }
}

% Font Awesome Icons (requires package fontawesome5)
\usepackage{fontawesome5} % For modern icons. Make sure to compile with XeLaTeX or LuaLaTeX if issues arise with regular LaTeX.
% You can add icons inside tcolorbox titles or itemize labels.
% Example: \faIcon[solid]{laptop-code} for a computer icon.

% Custom Commands
\newcommand{\objectivebox}[1]{%
    \begin{tcolorbox}[learningobjectives]
        \begin{itemize}
            #1
        \end{itemize}
    \end{tcolorbox}
}

\newcommand{\materialsbox}[1]{%
    \begin{tcolorbox}[materials]
        \begin{itemize}
            #1
        \end{itemize}
    \end{tcolorbox}
}

\newcommand{\lecturebox}[1]{%
    \begin{tcolorbox}[lecture]
        #1
    \end{tcolorbox}
}

\newcommand{\interactivebox}[1]{%
    \begin{tcolorbox}[interactive]
        #1
    \end{tcolorbox}
}

\setlength{\headheight}{15pt}

% For the program schedule table
\usepackage{longtable} % For tables that span multiple pages.
\usepackage{tabularx} % For tables with fixed width columns.
\usepackage{ragged2e} % For text alignment in table cells.
\usepackage{booktabs} % For professional table lines (toprule, midrule, bottomrule).
\usepackage{lscape} % For landscape orientation
\usepackage{rotating} % NEW: Required for rotating content like tables.
\usepackage{array}

% Custom table styles for consistency
\newcolumntype{L}[1]{>{\raggedright\arraybackslash}p{#1}} % Left-aligned column with fixed width.
\newcolumntype{C}[1]{>{\Centering\arraybackslash}p{#1}}   % Center-aligned column with fixed width.
\newcolumntype{R}[1]{>{\raggedleft\arraybackslash}p{#1}}   % Right-aligned column with fixed width.

% --- DOCUMENT-SPECIFIC MACROS ---
\newcommand{\programname}{Azebo Digital Foundation}
\newcommand{\docsubtitle}{Detailed Lesson Plan}
\newcommand{\teamname}{[Name]}
\newcommand{\docdate}{[Date]}

