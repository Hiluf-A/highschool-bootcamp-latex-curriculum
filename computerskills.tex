% computerskills.tex
% Content for the CS100 - Basic Computer Skills module.

\section{CS100 - Basic Computer Skills}
\addtocounter{section}{1}

\begin{tcolorbox}[boxstyle, title={Module Overview}]
   This module equips students with essential computer literacy skills required to thrive in today’s technology-driven world. It introduces fundamental concepts in computer hardware and software, operating system navigation, and cybersecurity best practices. Students will also develop practical competencies in touch typing through structured drills, along with hands-on experience in modern productivity applications and cloud-based collaboration tools. Emphasis is placed on consistent practice and real-world application to build confidence and efficiency in digital environments.
\end{tcolorbox}

\subsubsection*{Total Sessions: 6} % Use subsubsection or similar if not wanting it in ToC.

\subsection{S1: Introduction to Computers \& Operating Systems}
\objectivebox{
    \item \textbf{Distinguish} between various categories of computer hardware and software components.
    \item \textbf{Identify and articulate} the function of core computer hardware components, including the Central Processing Unit (CPU), Random Access Memory (RAM), and different types of storage devices (HDD, SSD).
    \item \textbf{Explain} the fundamental role and purpose of an Operating System (OS) as an intermediary between user applications and hardware.
    \item \textbf{Effectively navigate} and interact with the Graphical User Interface (GUI) of contemporary operating systems (e.g., Windows 11, macOS Sonoma).
    \item \textbf{Perform} basic OS operations such as launching and closing applications, accessing system information, and customizing desktop environments.
}
\materialsbox{
    \item \textbf{Hardware:} Desktop/laptop computers (1 per student), projector, interactive whiteboard.
    \item \textbf{Software:} Windows operating systems, common applications (web browser, text editor).
    \item \textbf{Teaching Aids:} Presentation slides, physical examples, animation on YouTube video, and detailed diagrams of computer internals (motherboard, CPU, RAM sticks), handouts with GUI navigation exercises.
}
\lecturebox{
    \begin{itemize}
        \item \textbf{Introduction to Computer Systems:} Definition of a computer, its evolution, and its ubiquitous role in modern society.
        \item \textbf{Hardware Components:} Detailed explanation of CPU (processing power), RAM (volatile memory for active tasks), Storage (persistent data storage – HDDs vs. SSDs), input/output devices. Visual aids will be used to deconstruct a computer.
        \item \textbf{Software Classification:} Distinction between System Software (Operating Systems, Utility Software) and Application Software (productivity, entertainment, specialized).
        \item \textbf{Operating Systems:} Core functions of an OS (resource management, process scheduling, memory management, file management, user interface). Comparison of popular OS types (Windows, macOS, Linux) and their use cases.
        \item \textbf{GUI Navigation:} Live demonstration of basic GUI elements: desktop, icons, taskbar/dock, start menu/applications folder, window management (minimize, maximize, close), basic settings access.
    \end{itemize}
}
\interactivebox{
    \begin{itemize}
        \item \textbf{Guided System Exploration (45 minutes):} Students will engage in a "system scavenger hunt" using a provided checklist. Tasks include: Locating and documenting their computer's CPU type, installed RAM, and primary storage capacity; Identifying the version of their operating system; Opening and closing at least three different pre-installed applications; Accessing and changing desktop background or theme settings; Connecting to a Wi-Fi network.
        \item \textbf{OS Comparison \& Discussion (45 minutes):} In small groups, students will be assigned different hypothetical scenarios (e.g., graphic design, gaming, software development) and discuss which operating system might be most suitable for each, justifying their choices. This will be followed by a class-wide discussion on the pros and cons of different OS environments.
        \item \textbf{Basic Troubleshooting Scenarios (30 minutes):} Instructor-led walk-through and student practice with simple common issues, such as force-quitting unresponsive applications using Task Manager (Windows) or Force Quit (macOS).
        \item \textbf{Assessment:} Completion of the scavenger hunt checklist, active participation in group discussions.
    \end{itemize}
}

\subsection{S2: Touch Typing Fundamentals}
\objectivebox{
    \item \textbf{Demonstrate} correct finger placement on the home row keys (ASDF JKL;) without visual aid of the keyboard.
    \item \textbf{Apply} ergonomic principles for optimal typing posture, hand, and wrist positioning to prevent strain and injury.
    \item \textbf{Practice and improve} typing accuracy and speed for basic words and sentences using a touch-typing methodology.
    \item \textbf{Identify} and correct common typing errors.
}
\materialsbox{
    \item \textbf{Hardware:} Computers with functional keyboards (standard QWERTY layout).
    \item \textbf{Software:} Access to reputable online typing tutors (TypingClub, Keybr.com, Ratatype and Typing.com).
    \item \textbf{Teaching Aids:} Diagrams illustrating correct ergonomic posture and hand placement, pre-designed practice sheets for offline reference.
}
\lecturebox{
    \begin{itemize}
        \item \textbf{Importance of Touch Typing:} Discussion on how touch typing significantly enhances productivity, reduces cognitive load, and promotes healthier work habits for digital professionals.
        \item \textbf{Ergonomics for Typists:} Detailed instruction and demonstration of proper seating posture, screen distance, arm and wrist positioning, and keyboard/mouse placement. Explanation of common repetitive strain injuries (RSIs) and how to prevent them.
        \item \textbf{Home Row Discipline:} Introduction to the QWERTY keyboard layout. Emphasis on the home row (ASDF JKL;) and the concept of returning fingers to these anchor keys.
        \item \textbf{Finger Mapping:} Demonstrating which fingers are responsible for which keys, expanding beyond the home row to adjacent keys.
        \item \textbf{Guided Practice Introduction:} Explanation of the chosen online typing platform's interface and progression.
    \end{itemize}
}
\interactivebox{
    \begin{itemize}
        \item \textbf{Initial Setup \& Ergonomic Check (15 minutes):} Students set up their workstations according to ergonomic guidelines. Instructors will circulate to provide individual adjustments and feedback on posture.
        \item \textbf{Guided Home Row Drills (45 minutes):} Students will begin with guided exercises on the chosen online platform, focusing exclusively on mastering the home row keys (ASDF JKL;) and the spacebar. They will practice recognizing key sounds and developing muscle memory.
        \item \textbf{Expanding Key Coverage (45 minutes):} Exercises will progressively introduce keys from the top row (QWERTY UIO) and bottom row (ZXCVBNM), emphasizing the correct finger assignment for each key. Timed drills will be incorporated to encourage speed development.
        \item \textbf{Accuracy Challenges \& Self-Correction (30 minutes):} Students will engage in exercises designed to improve accuracy, with immediate feedback from the typing tutor. They will be encouraged to self-correct errors and avoid looking at the keyboard.
        \item \textbf{Typing Games/Freestyle Practice (15 minutes):} A short period for students to engage in typing games on the platform or practice typing short paragraphs from a provided text, applying all learned techniques.
        \item \textbf{Assessment:} Students will be assessed through instructor observation (posture and finger placement), and performance metrics from an online typing tutor (accuracy and WPM). A first marked assessment worth 5\% will gauge early proficiency. A Fast Typer Competition will also be held to motivate progress and recognize top performers.








    \end{itemize}
}

\subsection{S3: File Management \& Organization}
\objectivebox{
    \item \textbf{Comprehend} the concept of a hierarchical file system and its importance for data organization.
    \item \textbf{Execute} fundamental file operations: creating, renaming, moving, copying, and deleting files and folders within an operating system's file explorer.
    \item \textbf{Identify and explain} the purpose of common file extensions (e.g., .docx, .pdf, .jpg, .py, .zip).
    \item \textbf{Differentiate} between local storage and cloud storage solutions, understanding their respective advantages and disadvantages.
    \item \textbf{Implement} logical naming conventions for files and folders to enhance searchability and maintainability.
}
\materialsbox{
    \item \textbf{Hardware:} Computers.
    \item \textbf{Software:} File Explorer (Windows), and cloud storage client (Google Drive).
    \item \textbf{Teaching Aids:} Presentation slides, pre-configured "disorganized" digital folders containing various file types, checklists for organizational tasks.
}
\lecturebox{
    \begin{itemize}
        \item \textbf{Digital Filing Cabinet Analogy:} Introduction to the concept of a file system as an organized structure for digital information, similar to physical filing cabinets.
        \item \textbf{Directories and Subdirectories:} Explanation of hierarchical structures (root, drives, folders, subfolders) and their role in logical data grouping.
        \item \textbf{File Operations:} Detailed demonstration of creating new folders, creating new files (e.g., text documents), renaming, cutting, copying, pasting, and deleting operations using the operating system's native tools. Emphasis on drag-and-drop vs. keyboard shortcuts.
        \item \textbf{File Extensions:} Explanation of common file extensions and their association with specific application types and data formats. Discussion on the importance of showing file extensions.
        \item \textbf{Local vs. Cloud Storage:} Comparative analysis of local hard drives vs. cloud-based solutions (Google Drive). Discussion of benefits (accessibility, backup, sharing) and considerations (security, privacy).
    \end{itemize}
}
\interactivebox{
    \begin{itemize}
        \item \textbf{Organizational Challenge (60 minutes):} Students are provided with a pre-configured, intentionally chaotic digital folder containing a mix of documents, images, videos, and various file types. Their task is to: Design and create a logical hierarchical folder structure; Move and rename existing files; Delete redundant files; Create at least three new files of different types.
        \item \textbf{Cloud Storage Practice (45 minutes):} Students will be guided to log into a pre-assigned or personal cloud storage account. They will: Upload a selection of files; Create a new folder; Share a specific file or folder with a partner.
        \item \textbf{File Extension Scavenger Hunt (15 minutes):} Students will be given a list of uncommon file extensions and tasked with using a search engine to find out what software typically opens them and what type of data they contain.
        \item \textbf{Assessment:} Instructor review of the students' organized file structures, successful upload/sharing of files to cloud storage, and completion of the file extension scavenger hunt.
    \end{itemize}
}

\subsection{S4: Internet \& Email Essentials}
\objectivebox{
   \item \textbf{Introduce} students to the basics of using the internet and web browsers, including navigating websites, understanding URLs, and using browser tools effectively.
\item \textbf{Execute} effective search engine queries utilizing advanced operators (e.g., quotation marks for exact phrases, \texttt{site:} for domain-specific searches, \texttt{filetype:} for specific document types).
\item \textbf{Critically evaluate} the credibility and reliability of online sources based on established criteria (e.g., authoritativeness, currency, objectivity, accuracy).
\item \textbf{Compose and send} professionally formatted emails, ensuring clear subject lines, appropriate greetings and closings, concise body paragraphs, and functional email signatures.
\item \textbf{Identify} and avoid common pitfalls in online research and email communication.

}
\materialsbox{
    \item \textbf{Hardware:} Computers with internet access.
    \item \textbf{Software:} Web browser (Chrome, Edge), email client (Gmail).
    \item \textbf{Teaching Aids:} Presentation slides, checklist for source evaluation, examples of professional and unprofessional emails, scenarios for research tasks.
}
\lecturebox{
    \begin{itemize}
        \item \textbf{Internet Architecture Basics:} A simplified overview of how the internet works (clients, servers, IP addresses, DNS).
        \item \textbf{Effective Search Strategies:} Beyond basic keywords: Exact phrase searches (" "), Exclusion of terms (-), Site-specific searches (\texttt{site:domain.com}), File type searches (\texttt{filetype:pdf}), Using \texttt{OR} for alternative terms, Boolean operators and wildcards.
        \item \textbf{Evaluating Online Sources:} Introduction to criteria for assessing credibility: author, publication date, domain type, bias, supporting evidence, cross-referencing. Discussion of fake news and misinformation.
        \item \textbf{Anatomy of a Professional Email:} Breakdown of essential components: Recipient (To, Cc, Bcc), Subject Line (concise and informative), Greeting, Body (clear, concise, direct), Call to Action (if any), Closing, Signature Block (professional contact info).
        \item \textbf{Email Etiquette:} Discussion on appropriate tone, urgency, attachment handling, and reply discipline.
    \end{itemize}
}
\interactivebox{
    \begin{itemize}
     \item \textbf{Typing Practice Integration (30 minutes):} Dedicated time for students to continue practicing their touch-typing skills using the online platform.
        \item \textbf{Advanced Search Challenge (45 minutes):} In small groups, students will be given 2-3 specific research topics. For each topic, they must: Formulate at least three different search queries using advanced operators; Identify and select two credible online sources, justifying their credibility; Identify one non-credible source and explain why.
        \item \textbf{Professional Email Drafting (45 minutes):} Each student will draft two distinct emails: 1) A formal email to the instructor summarizing research findings; 2) A scenario-based email. Students will exchange drafts for peer review.
        \item \textbf{Assessment:} Quality of search queries and justification of source credibility during group presentation, adherence to professional email formatting and content guidelines in drafted emails.
    \end{itemize}
}

\subsection{S5: Cybersecurity Basics}
\objectivebox{
    \item \textbf{Define} common cybersecurity threats, including malware (viruses, ransomware), phishing, and social engineering.
    \item \textbf{Formulate} strong, unique passwords adhering to current best practices (e.g., length, complexity, randomness, avoidance of personal information).
    \item \textbf{Identify} key characteristics and red flags commonly found in phishing attempts (e.g., suspicious links, urgent/threatening language, grammatical errors).
    \item \textbf{Explain} the function and importance of Two-Factor Authentication (2FA) in enhancing account security.
    \item \textbf{Develop} an awareness of personal data privacy and the importance of secure online habits.
}
\materialsbox{
    \item \textbf{Hardware:} Computers.
    \item \textbf{Software:} Web browser, password strength checker tool (online), simulated phishing email examples.
    \item \textbf{Teaching Aids:} Presentation slides, infographics on password best practices, examples of real and fake phishing emails, checklist for identifying phishing.
}
\lecturebox{
    \begin{itemize}
        \item \textbf{Introduction to Cybersecurity:} Definition of cybersecurity and its relevance in daily life. Overview of the "CIA Triad" (Confidentiality, Integrity, Availability) as core principles.
        \item \textbf{Common Cyber Threats:} Malware (viruses, worms, trojans, ransomware), Phishing (tactics, targets), Social Engineering (manipulation tactics).
        \item \textbf{Strong Passwords:} Dispelling myths, password entropy, recommended length, character mix, unique passwords, password managers.
        \item \textbf{Two-Factor Authentication (2FA):} Explanation of "something you know" + "something you have/are." Demonstration of how 2FA works.
        \item \textbf{Personal Data Privacy:} What personal data is, why it's valuable, strategies for protecting it online.
    \end{itemize}
}
\interactivebox{
    \begin{itemize}
         \item \textbf{Typing Practice Integration (30 minutes):} Dedicated time for students to continue practicing their touch-typing skills using the online platform.
        \item \textbf{Password Creation Workshop (30 minutes):} Students use an online password strength checker to test strategies and collaboratively develop strong passwords. Discussion on remembering unique passwords.
        \item \textbf{Phishing Email Analysis (45 minutes):} Students analyze sample emails (legitimate/phishing) to identify red flags and categorize them. Class discussion on "phishing literacy."
        \item \textbf{2FA Simulation \& Discussion (15 minutes):} Walkthrough of 2FA setup on a common platform. Discussion of scenarios where 2FA prevents breaches.
        \item \textbf{Assessment:} Participation in password workshop, ability to formulate strong passwords, accurate identification of phishing attempts.
    \end{itemize}
}

\subsection{S6: Productivity \& Cloud Collaboration}
\objectivebox{
    \item \textbf{Create and format} basic text documents using a word processing application (Google Docs, Microsoft Word), incorporating elements like headings, lists, and basic text styling.
    \item \textbf{Develop and manipulate} simple spreadsheets (Google Sheets, Microsoft Excel), including data entry, basic arithmetic calculations (sum, average), and chart creation.
    \item \textbf{Utilize} cloud-based tools for real-time collaborative document editing, commenting, and version control.
    \item \textbf{Demonstrate} effective teamwork and communication within a shared digital workspace.
}
\materialsbox{
    \item \textbf{Hardware:} Computers with internet access.
    \item \textbf{Software:} Access to Google Workspace (Docs, Sheets) or Microsoft 365 (Word, Excel) online.
    \item \textbf{Teaching Aids:} Presentation slides, sample documents and spreadsheets for collaborative tasks, collaboration checklist.
}
\lecturebox{
    \begin{itemize}
        \item \textbf{Introduction to Productivity Suites:} Overview of integrated software suites for office tasks.
        \item \textbf{Word Processors - Core Functions:} Creating, opening, saving documents. Basic formatting (font size/style, bold, italics, underline), paragraph alignment, lists, headings, inserting simple images.
        \item \textbf{Spreadsheets - Core Functions:} Cells, rows, columns. Data entry, basic data types. Simple formulas: \texttt{SUM()}, \texttt{AVERAGE()}, cell referencing. Creating simple charts.
        \item \textbf{Cloud Collaboration Principles:} Why cloud tools are essential for teamwork. Features like real-time co-editing, commenting, suggesting edits, version history, sharing permissions.
        \item \textbf{Best Practices for Collaboration:} Communication within collaborative documents, assigning tasks, managing conflicts.
    \end{itemize}
}
\interactivebox{
    \begin{itemize}
         \item \textbf{Typing Practice Integration (30 minutes):} Dedicated time for students to continue practicing their touch-typing skills using the online platform.
        \item \textbf{Collaborative Document Creation (60 minutes):} Students work in pairs/groups on a shared document. Task: Research a topic, collaboratively draft a short report using formatting, commenting, and suggested edits.
        \item \textbf{Shared Spreadsheet \& Data Analysis (60 minutes):} Using a shared spreadsheet, students enter mock data, apply basic formulas (\texttt{SUM}, \texttt{AVERAGE}), and collaboratively create a simple chart.
        \item \textbf{Assessment:} Instructor observation of collaborative workflow, proper use of document/spreadsheet features, quality of formatted documents/calculated spreadsheets.
    \end{itemize}
}
