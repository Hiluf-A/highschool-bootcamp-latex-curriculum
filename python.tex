% python.tex
% Content for the PY101 - Introduction to Computer Programming with Python module.

\section{PY101 - Introduction to Computer Programming with Python}
\addtocounter{section}{3} % Ensure section counter is correct if needed for auto-numbering consistency.

\begin{tcolorbox}[boxstyle, title={Module Overview}]
    This module introduces students to the fundamental concepts of computer programming using the Python language, covering basic syntax and culminating in a mentored mini-project. The focus is on exploring the core principles of programming and demonstrating what students can achieve by developing coding skills. Through hands-on exercises and problem-solving activities, students will gain an introductory understanding of programming logic and its real-world applications, setting a foundation for future learning.
\end{tcolorbox}

\subsubsection*{Total Sessions: 9}

\begin{tcolorbox}[boxstyle, title={Session Structure}]
    Each session maintains the 1-hour lecture + 2-hour interactive format:
    \begin{itemize}
        \item \textbf{Lectures} introduce programming fundamentals through conceptual explanations, Python syntax demonstrations, and exploratory discussions of programming possibilities.
        
        \item \textbf{Interactive Sessions} feature guided coding explorations, collaborative concept-mapping exercises, and scaffolded problem-solving activities. Students discover programming applications through carefully structured challenges of progressive complexity. Project-based sessions (S8, S9) will operate as discovery workshops where instructors facilitate conceptual understanding, demonstrate solution pathways, and highlight creative potential through mentor-guided experimentation.
    \end{itemize}
\end{tcolorbox}

% --- Example detailed sessions for PY101 (expanding on the overview) ---

\subsection{S1: Intro to Python}
\objectivebox{
    \item \textbf{Recognize} the core components of a Python program.
    \item \textbf{Observe} program execution through simple output examples.
    \item \textbf{Identify} fundamental data representations (integers, floats, strings, booleans).
    \item \textbf{Understand} how variables act as containers for data.
}
\materialsbox{
    \item \textbf{Hardware:} Computers.
    \item \textbf{Software:} Python 3 interpreter, IDE (e.g., VS Code) or online interpreter.
    \item \textbf{Teaching Aids:} Conceptual diagrams, exploratory code snippets, "Why Programming?" discussion prompts.
}
\lecturebox{
    \begin{itemize}
        \item \textbf{The Landscape of Programming:} How algorithms solve real-world problems. Python's role in data science, automation, and creative coding.
        \item \textbf{Environment as a Discovery Tool:} Brief environment setup emphasizing its purpose for exploration.
        \item \textbf{"Hello World!" as Concept Demonstration:} Using \texttt{print()} to reveal program flow.
        \item \textbf{Data Representation Concepts:} How computers model information through types. Real-world analogies (e.g., numbers as measurements, text as messages).
        \item \textbf{Variables as Labels:} Metaphors for assignment operations. Basic arithmetic as transformation examples.
    \end{itemize}
}
\interactivebox{
    \begin{itemize}
        \item \textbf{Environment Exploration (30 minutes):} Guided discovery of coding interfaces. Observe execution of provided "Hello World!" examples.
        
        \item \textbf{Data Type Investigation (45 minutes):} 
        \begin{itemize}
            \item Modify pre-written variable examples to observe type behaviors
            \item Experiment with \texttt{type()} to discover data categories
        \end{itemize}
        
        \item \textbf{Variable Relationships (45 minutes):} 
        \begin{itemize}
            \item Adjust given arithmetic templates to observe computational outcomes
            \item Mapping exercise: Connect variables to real-world representations (e.g., \texttt{temperature = 28.5})
        \end{itemize}
        
        \item \textbf{Concept Synthesis (30 minutes):} 
        \begin{itemize}
            \item Group discussion: "How might these concepts enable creating a simple calculator?"
            \item Reflection: "Where have you encountered similar data representations?"
        \end{itemize}
        
        \item \textbf{Assessment:} Participation in explorations, accurate observations of type behaviors, conceptual connections during synthesis.
    \end{itemize}
}
\subsection{S2: Operators \& Input}
\objectivebox{
    \item \textbf{Recognize} different operator categories and their purposes in expressions.
    \item \textbf{Observe} how user interaction works through the `input()` function.
    \item \textbf{Understand} the concept of data type conversion in program interactions.
    \item \textbf{Explore} how simple programs can respond to user-provided information.
}
\materialsbox{
    \item \textbf{Hardware:} Computers.
    \item \textbf{Software:} Python 3 interpreter, IDE/online interpreter.
    \item \textbf{Teaching Aids:} Concept diagrams showing operator functions, real-world analogies for input/output, sample conversational programs.
}
\lecturebox{
    \begin{itemize}
        \item \textbf{Operators as Building Blocks:} Conceptual overview of arithmetic, comparison, and logical operators as tools for expressing relationships.
        \item \textbf{Programs as Conversations:} Metaphor of `input()` as "asking questions" and `print()` as "giving answers".
        \item \textbf{Data Types in Dialog:} Why input needs conversion (text vs. numbers) demonstrated through relatable examples.
        \item \textbf{Simple Decision Concepts:} Introduction to boolean results as yes/no outcomes using everyday comparisons.
    \end{itemize}
}
\interactivebox{
    \begin{itemize}
        \item \textbf{Operator Exploration (45 minutes):}
        \begin{itemize}
            \item Experiment with provided operator templates to observe outcomes
            \item Predict-and-check exercises with comparison chains (e.g., `5 > 3 and 2 == 2`)
        \end{itemize}
        
        \item \textbf{Input Discovery (45 minutes):}
        \begin{itemize}
            \item Run pre-built programs that use `input()` for simple questionnaires
            \item Modify conversion functions (`int()`, `float()`) to observe type changes
        \end{itemize}
        
        \item \textbf{Concept Connection (30 minutes):}
        \begin{itemize}
            \item Guided creation of a number-guessing hint generator (e.g., "Is your number > 10?")
            \item Discussion: "How could these concepts enable a quiz program?"
        \end{itemize}
        
        \item \textbf{Assessment:} Participation in explorations, accurate predictions of operator behaviors, understanding demonstrated in concept discussions.
    \end{itemize}
}

% ... (Continue with detailed plans for S3-S9 following the same structure) ...
% For brevity, only S1 and S2 are detailed here as examples.
% The actual PY101.tex file should include S3, S4, S5, S6, S7, S8, S9 details.

\subsection{S3: Conditionals}
\objectivebox{
    \item \textbf{Recognize} the concept of decision-making and branching logic in programs.
    \item \textbf{Observe} the flow of program execution based on conditions using \texttt{if}, \texttt{elif}, and \texttt{else} statements.
    \item \textbf{Understand} how boolean expressions control different program paths.
    \item \textbf{Explore} how to build programs that respond differently to varying inputs and scenarios.
}
\materialsbox{
    \item \textbf{Hardware:} Computers.
    \item \textbf{Software:} Python 3 interpreter, IDE/online interpreter.
    \item \textbf{Teaching Aids:} Flowcharts illustrating conditional logic, scenario cards for decision-making programs (e.g., a "choose your own adventure" game snippet), debugging examples with common logical errors.
}
\lecturebox{
    \begin{itemize}
        \item \textbf{Programs as Decision-Makers:} Introduce the idea that programs need to make choices, mimicking human decision processes. Analogies: traffic lights responding to cars, choosing clothes based on weather conditions.
        \item \textbf{The \texttt{if} Statement:} Basic syntax and the concept of executing a block of code \textit{only if} a specified condition evaluates to \texttt{True}.
        \item \textbf{\texttt{else} for Alternatives:} Demonstrating how the \texttt{else} block provides a fallback path when the initial \texttt{if} condition evaluates to \texttt{False}.
        \item \textbf{\texttt{elif} for Multiple Choices:} Handling scenarios with more than two possible outcomes, explaining the order of evaluation for \texttt{elif} statements.
        \item \textbf{Boolean Expressions Revisited:} Deepen understanding of how comparison and logical operators combine to form the complex conditions that \texttt{if} statements evaluate, leading to \texttt{True} or \texttt{False} outcomes.
    \end{itemize}
}
\interactivebox{
    \begin{itemize}
        \item \textbf{Basic \texttt{if} Statement Exploration (45 minutes):}
        \begin{itemize}
            \item Run and modify provided simple code snippets containing \texttt{if} statements (e.g., checking if a number is positive, if a string is empty).
            \item Predict program outcomes for different input values, focusing on whether the conditional block executes.
        \end{itemize}

        \item \textbf{\texttt{if-else} and \texttt{elif} Challenges (60 minutes):}
        \begin{itemize}
            \item Develop programs that use \texttt{if-else} for binary choices (e.g., checking voting eligibility based on age, determining pass/fail).
            \item Create programs incorporating \texttt{elif} for multiple choices (e.g., a simple grading system assigning A, B, C, D, or F based on a score range).
            \item Introduce and explore nested conditionals for more complex decision trees (e.g., checking both age and if a special ID is present for entry).
        \end{itemize}

        \item \textbf{Debugging Conditional Logic (30 minutes):}
        \begin{itemize}
            \item Students will be presented with programs containing subtle logic errors within conditionals (e.g., incorrect operator, wrong order of \texttt{elif} clauses, off-by-one errors in ranges).
            \item Guided practice in using print statements or a debugger to identify and rectify these logical flaws.
        \end{itemize}

        \item \textbf{Concept Synthesis (15 minutes):}
        \begin{itemize}
            \item Group discussion: "How can conditionals make programs appear 'intelligent' or more 'responsive' to user input?"
            \item Brainstorming real-world scenarios where conditionals are indispensable (e.g., simple games, login systems, automated responses).
        \end{itemize}

        \item \textbf{Assessment:} Successful completion of coding challenges involving \texttt{if}, \texttt{elif}, and \texttt{else}, demonstrated ability to identify and fix conditional logic errors, and active participation in concept discussions.
    \end{itemize}
}
\subsection{S4: Loops}
\objectivebox{
    \item \textbf{Recognize} the need for repetition in programming.
    \item \textbf{Understand} the concept of iteration using \texttt{for} and \texttt{while} loops.
    \item \textbf{Observe} how loops automate repetitive tasks.
    \item \textbf{Explore} how to control loop execution (e.g., \texttt{break}, \texttt{continue}).
}
\materialsbox{
    \item \textbf{Hardware:} Computers.
    \item \textbf{Software:} Python 3 interpreter, IDE/online interpreter.
    \item \textbf{Teaching Aids:} Diagrams illustrating loop flow (flowcharts), examples of repetitive real-world tasks, debugging examples for infinite loops.
}
\lecturebox{
    \begin{itemize}
        \item \textbf{The Power of Repetition:} Introduce the concept of loops as a way to avoid writing repetitive code. Analogies: a recipe instructing to stir until smooth, a fitness routine repeating exercises.
        \item \textbf{\texttt{for} loops (Iterating over Sequences):} Basic syntax of \texttt{for} loops. Explanation of iterating over strings, lists (briefly introduce lists conceptually as sequences), and \texttt{range()} function.
        \item \textbf{\texttt{while} loops (Conditional Repetition):} Syntax of \texttt{while} loops. Concept of a loop continuing as long as a condition is \texttt{True}. Importance of loop termination to avoid infinite loops.
        \item \textbf{Controlling Loop Flow:} \texttt{break} statement (exiting a loop early). \texttt{continue} statement (skipping current iteration).
    \end{itemize}
}
\interactivebox{
    \begin{itemize}
        \item \textbf{\texttt{for} Loop Exploration (50 minutes):}
        \begin{itemize}
            \item Run and modify simple \texttt{for} loops to print sequences of numbers or characters.
            \item Practice iterating over strings to count specific characters.
            \item Create a program that prints a multiplication table for a given number using a \texttt{for} loop and \texttt{range()}.
        \end{itemize}

        \item \textbf{\texttt{while} Loop Challenges (50 minutes):}
        \begin{itemize}
            \item Develop a program using a \texttt{while} loop to count down from a number to zero.
            \item Create a simple interactive program that repeatedly asks for user input until a specific keyword is entered (e.g., "quit").
            \item Introduce a simple guessing game where the user guesses a number until correct, using a \texttt{while} loop.
        \end{itemize}

        \item \textbf{Loop Control Statements Practice (30 minutes):}
        \begin{itemize}
            \item Implement \texttt{break} in a loop that searches for a specific item in a list and stops once found.
            \item Implement \texttt{continue} in a loop that prints numbers but skips multiples of 3.
            \item Debugging: Introduce an infinite \texttt{while} loop and guide students to identify and fix the termination condition.
        \end{itemize}

        \item \textbf{Concept Synthesis (10 minutes):}
        \begin{itemize}
            \item Group discussion: "When would you choose a \texttt{for} loop versus a \texttt{while} loop?"
            \item Brainstorming real-world applications of loops (e.g., processing data, game animations, repetitive calculations).
        \end{itemize}

        \item \textbf{Assessment:} Successful completion of coding challenges using both \texttt{for} and \texttt{while} loops, correct application of \texttt{break} and \texttt{continue}, demonstrated ability to prevent/debug infinite loops, and active participation in concept discussions.
    \end{itemize}
}
\subsection{S5: Functions}
\objectivebox{
    \item \textbf{Recognize} the concept of code reusability and modularity through functions.
    \item \textbf{Understand} how to define and call basic functions in Python.
    \item \textbf{Observe} the role of parameters and return values in functions.
    \item \textbf{Explore} how functions help organize code and solve larger problems incrementally.
}
\materialsbox{
    \item \textbf{Hardware:} Computers.
    \item \textbf{Software:} Python 3 interpreter, IDE/online interpreter.
    \item \textbf{Teaching Aids:} Diagrams illustrating function calls and data flow, analogy of functions as "mini-programs" or "recipes," code examples demonstrating function definition and calls.
}
\lecturebox{
    \begin{itemize}
        \item \textbf{The Need for Functions:} Introduce the idea of avoiding "copy-pasting" repetitive code. Analogy: a machine that performs a specific task.
        \item \textbf{Defining Functions (\texttt{def}):} Basic syntax for defining a function using the \texttt{def} keyword, function name, parentheses, and colon. Importance of indentation for the function body.
        \item \textbf{Calling Functions:} How to execute a defined function.
        \item \textbf{Parameters (Inputs to Functions):} Explain how parameters act as placeholders for values passed into a function, allowing it to be flexible.
        \item \textbf{Return Values (Outputs from Functions):} The \texttt{return} keyword and its role in sending a result back from a function. Distinction between printing and returning.
        \item \textbf{Scope (Brief Introduction):} Briefly touch upon local variables within functions vs. global variables outside, without deep dive.
    \end{itemize}
}
\interactivebox{
    \begin{itemize}
        \item \textbf{Basic Function Definition and Calling (50 minutes):}
        \begin{itemize}
            \item Write simple functions that greet a user or perform a basic calculation without parameters or return values initially.
            \item Practice calling these functions multiple times to observe code reuse.
            \item Modify a previous loop or conditional program to use a simple function to perform a repeated action.
        \end{itemize}

        \item \textbf{Functions with Parameters and Return Values (50 minutes):}
        \begin{itemize}
            \item Create functions that take one or more parameters (e.g., a function that calculates the area of a rectangle, taking length and width).
            \item Implement functions that return a value (e.g., a function that returns the result of adding two numbers, or a function that returns a formatted string).
            \item Combine concepts: create a function that takes input, performs a calculation using parameters, and returns the result.
        \end{itemize}

        \item \textbf{Modular Problem Solving (30 minutes):}
        \begin{itemize}
            \item Take a slightly larger problem (e.g., a simple text-based "calculator" that adds, subtracts, etc.) and break it down into smaller functions.
            \item Discuss how functions improve readability and make debugging easier.
            \item Debugging: Introduce functions with incorrect parameter passing or missing return statements, and guide students to fix them.
        \end{itemize}

        \item \textbf{Concept Synthesis (10 minutes):}
        \begin{itemize}
            \item Group discussion: "How do functions help us build more complex programs?"
            \item Brainstorming real-world parallels for functions (e.g., a vending machine, a button on a remote control).
        \end{itemize}

        \item \textbf{Assessment:} Successful definition and calling of functions with and without parameters/return values, demonstrated understanding of modularity, and active participation in discussions on problem decomposition.
    \end{itemize}
}

\subsection{\texorpdfstring{S6 \& S7:\\ Data Structures}{S6 \& S7: Data Structures}}

\subsubsection{S6: Data Structures 1 (Lists, Tuples)}
\objectivebox{
    \item \textbf{Recognize} the concept of collections for storing multiple pieces of data.
    \item \textbf{Understand} the characteristics and common uses of Python Lists.
    \item \textbf{Observe} how to create, access, modify, and iterate through Lists.
    \item \textbf{Explore} the immutability and applications of Python Tuples.
}
\materialsbox{
    \item \textbf{Hardware:} Computers.
    \item \textbf{Software:} Python 3 interpreter, IDE/online interpreter.
    \item \textbf{Teaching Aids:} Visual diagrams of lists/tuples, analogies (e.g., shopping list, fixed recipe ingredients), common list/tuple methods cheatsheet.
}
\lecturebox{
    \begin{itemize}
        \item \textbf{Beyond Single Variables:} Introduce the idea of needing to store collections of related data. Analogy: a single box vs. a shelving unit.
        \item \textbf{Lists (\texttt{[]}):}
            \begin{itemize}
                \item Definition: Ordered, mutable collections.
                \item Creation: empty lists, lists with initial elements.
                \item Accessing Elements: indexing (positive and negative), slicing.
                \item Modifying Lists: adding (append, insert, extend), removing (remove, pop, del), changing elements.
                \item Common List Methods: \texttt{len()}, \texttt{sort()}, \texttt{count()}, \texttt{index()}.
            \end{itemize}
        \item \textbf{Tuples (\texttt{()}):}
            \begin{itemize}
                \item Definition: Ordered, immutable collections.
                \item Creation: tuple literals, single-element tuples.
                \item Accessing Elements: indexing, slicing (similar to lists).
                \item Immutability: demonstrating what cannot be changed in a tuple.
                \item Use Cases: When data should not change (e.g., coordinates, record of events).
            \end{itemize}
        \item \textbf{Iterating with Loops:} Briefly revisit \texttt{for} loops to iterate over elements in lists and tuples.
    \end{itemize}
}
\interactivebox{
    \begin{itemize}
        \item \textbf{List Manipulation Exercises (60 minutes):}
        \begin{itemize}
            \item Create lists of different data types.
            \item Practice adding/removing elements (e.g., manage a to-do list, update student names).
            \item Use indexing and slicing to retrieve specific parts of a list.
            \item Write a program to find the largest/smallest number in a list.
        \end{itemize}

        \item \textbf{Tuple Exploration (45 minutes):}
        \begin{itemize}
            \item Create tuples and attempt to modify them to observe immutability errors.
            \item Use tuples to store fixed data (e.g., a person's birthdate, RGB color values).
            \item Convert between lists and tuples.
        \end{itemize}

        \item \textbf{Combining Concepts (25 minutes):}
        \begin{itemize}
            \item Write a function that takes a list of numbers and returns their average.
            \item Use a loop to process elements from a list of strings.
        \end{itemize}

        \item \textbf{Assessment:} Successful creation and manipulation of lists, demonstrated understanding of tuple immutability, correct use of list/tuple methods in coding exercises.
    \end{itemize}
}

\subsubsection{S7: Data Structures 2 (Dictionaries, Sets)}
\objectivebox{
    \item \textbf{Understand} the concept of key-value pairs and their application in Dictionaries.
    \item \textbf{Recognize} the characteristics and common uses of Python Dictionaries.
    \item \textbf{Observe} how to create, access, modify, and iterate through Dictionaries.
    \item \textbf{Explore} the concept of unique elements and their application in Python Sets.
}
\materialsbox{
    \item \textbf{Hardware:} Computers.
    \item \textbf{Software:} Python 3 interpreter, IDE/online interpreter.
    \item \textbf{Teaching Aids:} Analogies for dictionaries (e.g., phonebook, glossary), analogies for sets (e.g., unique collection of items), common dictionary/set methods cheatsheet.
}
\lecturebox{
    \begin{itemize}
        \item \textbf{Dictionaries (\texttt{\{\}}):}
            \begin{itemize}
                \item Definition: Unordered, mutable collections of key-value pairs. Keys must be unique and immutable.
                \item Creation: Empty dictionaries, dictionaries with initial pairs.
                \item Accessing Elements: using keys. Handling missing keys.
                \item Modifying Dictionaries: adding new pairs, updating values, removing pairs (\texttt{del}, \texttt{pop}).
                \item Common Dictionary Methods: \texttt{keys()}, \texttt{values()}, \texttt{items()}, \texttt{get()}.
                \item Use Cases: Storing structured data (e.g., student records, configuration settings).
            \end{itemize}
        \item \textbf{Sets (\texttt{\{\}}):}
            \begin{itemize}
                \item Definition: Unordered collections of unique elements. Useful for membership testing and eliminating duplicates.
                \item Creation: from lists, set literals.
                \item Basic Set Operations: adding, removing elements.
                \item Mathematical Set Operations: union, intersection, difference, symmetric difference.
                \item Use Cases: Checking for unique items, finding common elements between collections.
            \end{itemize}
        \item \textbf{Choosing the Right Data Structure:} Brief discussion on when to use lists, tuples, dictionaries, or sets based on problem requirements.
    \end{itemize}
}
\interactivebox{
    \begin{itemize}
        \item \textbf{Dictionary Manipulation Exercises (60 minutes):}
        \begin{itemize}
            \item Create a dictionary to store contact information (name, phone, email).
            \item Practice adding new contacts, updating phone numbers, and deleting contacts.
            \item Write a program to search for a key or value in a dictionary.
            \item Iterate through dictionary keys, values, and items using loops.
        \end{itemize}

        \item \textbf{Set Operations Practice (45 minutes):}
        \begin{itemize}
            \item Create sets from lists to remove duplicate elements.
            \item Perform union, intersection, and difference operations between two sets (e.g., finding common hobbies between two people).
            \item Use sets for fast membership checking.
        \end{itemize}

        \item \textbf{Data Structure Application Challenge (25 minutes):}
        \begin{itemize}
            \item A mini-challenge combining multiple data structures (e.g., using a list of dictionaries to store multiple student records, then using sets to find unique courses taken).
            \item Discuss problem-solving approaches for data storage.
        \end{itemize}

        \item \textbf{Assessment:} Successful creation and manipulation of dictionaries, correct application of set operations, ability to choose appropriate data structures for given problems.
    \end{itemize}
}


\subsection{\texorpdfstring{S8: Project Session \\ Simple Calculator}
                              {S8: Project Session Simple Calculator}}

\objectivebox{
    \item \textbf{Apply} all learned Python concepts (input, variables, operators, conditionals, functions) to build a cohesive application.
    \item \textbf{Design} a user-friendly command-line interface for interaction.
    \item \textbf{Implement} error handling for invalid user inputs (e.g., non-numeric input, division by zero).
    \item \textbf{Practice} modular programming by breaking down the calculator's functionality into distinct functions.
}
\materialsbox{
    \item \textbf{Hardware:} Computers.
    \item \textbf{Software:} Python 3 interpreter, IDE/online interpreter.
    \item \textbf{Teaching Aids:} Flowchart examples for calculator logic.
}
\lecturebox{
    \begin{itemize}
        \item \textbf{Project Introduction \& Scope (15 minutes):} Define the simple calculator project. Discuss its components (getting numbers, choosing operation, performing calculation, displaying result).
        \item \textbf{Modular Design for Calculators (20 minutes):} Review how to break down the calculator into functions: \texttt{add(a, b)}, \texttt{subtract(a, b)}, \texttt{multiply(a, b)}, \texttt{divide(a, b)}. Emphasize main loop for user interaction.
        \item \textbf{Error Handling Fundamentals (15 minutes):} Introduction to common errors (e.g., \texttt{ValueError} for non-numeric input, \texttt{ZeroDivisionError}). Simple \texttt{try-except} blocks to gracefully handle these.
        \item \textbf{User Experience Considerations (10 minutes):} Tips for clear prompts, informative messages, and a continuous loop until the user decides to quit.
    \end{itemize}
}
\interactivebox{
    \begin{itemize}
        \item \textbf{Core Calculator Logic (60 minutes):}
        \begin{itemize}
            \item Students start by implementing the four basic arithmetic functions (\texttt{add}, \texttt{subtract}, \texttt{multiply}, \texttt{divide}).
            \item Guided practice in getting numeric input from the user and converting it.
            \item Implement conditional logic to select the correct operation based on user input (e.g., "+", "-", "*", "/").
        \end{itemize}

        \item \textbf{User Interface and Loop (50 minutes):}
        \begin{itemize}
            \item Wrap the core logic in a \texttt{while} loop to allow multiple calculations without restarting the program.
            \item Implement a "quit" option for the user.
            \item Focus on clear \texttt{print()} statements for instructions and results.
        \end{itemize}

        \item \textbf{Error Handling Implementation (30 minutes, as a challenge):}
        \begin{itemize}
            \item Add \texttt{try-except} blocks around input conversion to handle non-numeric input.
            \item Implement specific handling for \texttt{ZeroDivisionError} in the division function.
            \item Test the calculator with various invalid inputs to ensure robust behavior.
        \end{itemize}

        \item \textbf{Assessment:} A functional simple calculator program, clear and concise code, proper use of functions and conditionals, basic error handling implemented.
    \end{itemize}
}

\subsection{\texorpdfstring{S9: Project Session\\ Number Guessing Game} {S9: Project Session Number Guessing Game}}

\objectivebox{
    \item \textbf{Utilize} random number generation to create unpredictable game elements.
    \item \textbf{Implement} a game loop that continues until a winning condition is met.
    \item \textbf{Apply} conditional logic to provide user feedback (e.g., "too high," "too low").
    \item \textbf{Track} and display the number of attempts the user takes to guess correctly.
}
\materialsbox{
    \item \textbf{Hardware:} Computers.
    \item \textbf{Software:} Python 3 interpreter, IDE/online interpreter.
    \item \textbf{Teaching Aids:} Diagram illustrating the game flow, `random` module documentation (simplified), examples of game feedback messages.
}
\lecturebox{
    \begin{itemize}
        \item \textbf{Game Introduction \& Scope (15 minutes):} Define the number guessing game. Outline its rules (computer picks, user guesses, hints provided).
        \item \textbf{Random Number Generation (\texttt{random} module) (20 minutes):} Introduce the \texttt{random} module, specifically \texttt{random.randint()} for picking a secret number within a range.
        \item \textbf{The Game Loop (\texttt{while} loop) (15 minutes):} Explain how a \texttt{while} loop is perfect for this game, continuing as long as the guess is incorrect.
        \item \textbf{Conditional Feedback (10 minutes):} Discuss how \texttt{if}-\texttt{elif}-\texttt{else} statements will be used to tell the user if their guess is too high, too low, or correct.
        \item \textbf{Counting Attempts (5 minutes):} Show how a simple counter variable can track the number of guesses.
    \end{itemize}
}
\interactivebox{
    \begin{itemize}
        \item \textbf{Setting Up the Game (60 minutes):}
        \begin{itemize}
            \item Import the \texttt{random} module and generate a random secret number (e.g., between 1 and 100).
            \item Initialize a counter for guesses.
            \item Create the basic game loop using \texttt{while False} or a placeholder condition that will be updated.
            \item Get the first user guess.
        \end{itemize}

        \item \textbf{Implementing Game Logic (50 minutes):}
        \begin{itemize}
            \item Inside the loop, use \texttt{if}-\texttt{elif}-\texttt{else} to compare the user's guess with the secret number.
            \item Provide appropriate feedback: "Too high!", "Too low!", "Correct!".
            \item Update the counter with each guess.
            \item Adjust the \texttt{while} loop condition so it terminates when the correct number is guessed.
        \end{itemize}

        \item \textbf{Refinement and Polish (30 minutes):}
        \begin{itemize}
            \item Add a message displaying the number of attempts when the user wins.
            \item Implement error handling for non-numeric input (reusing skills from S8).
            \item Optional challenge: Allow the user to choose the range of the secret number.
            \item Test the game thoroughly with various guesses and edge cases.
        \end{itemize}

        \item \textbf{Assessment:} A fully functional number guessing game, correct use of random numbers, effective game loop, appropriate conditional feedback, and accurate tracking of attempts.
    \end{itemize}
}
